\section{Discussion}

Over the last few decades, there have been tremendous efforts in understanding how biology produces robust reproducible patterns, e.g. during embryonic development, with seminal work by Turing and others (\cite{Turing1952,Gierer1972, maini2012turing}). Robustness, sometimes called structural robustness, refers to the fraction of parameter space leading to Turing patterns. Due to the Turing conditions, this robustness is generally tiny, hindering progress in understanding developmental patterns. Here, we investigate how a highly nonlinear reaction-diffusion model motivated by synthetic gene circuits can generate spatio-temporal patterns beyond predictions from linear stability analysis (LSA). The discrepancies between LSA and numerical predictions are considerable and can be attributed to non-linearities, boundary conditions, and growth. A specific focus of our work is the role of multistability on Turing systems, which arises from nonlinearities and feedbacks loops. We describe the mechanism by which an unstable system can acquire Turing patterns, and conversely how Turing states can lose their patterns. These findings make the Turing mechanism significantly more versatile than originally assumed, and point towards investigating the effects of other biological properties such as absorbing boundary conditions or growing domains on patterning robustness. 

Most current robustness studies primarily focus on Turing I instabilities ~\parencite{Scholes2019, Zheng2016, Marcon}.
However, we demonstrate that numerical methods for robustness searches can uncover a wider variety of relevant spatio-temporal solutions.
For example, systems exhibiting unstable, Hopf or Turing I-Hopf dispersion relations can also produce Turing-like stationary periodic patterns (see Fig.~\ref{fig:dispersions}F). Therefore, systems with such dispersion relations should not be disregarded when studying pattern formation.
Although considering these systems might slightly improve robustness, it does not fully explain the difference between robust mechanisms found in nature and the non-robust Turing patterns. The latter still require fine-tuning due to mathematical constraints.
Additionally, some systems with a Hopf-dispersion relation produce noteworthy non-stationary patterns.
These non-stationary patterns could be highly valuable in developmental and synthetic biology because arrest of gene expression could transform them into stationary patterns.
Furthermore, these non-stationary patterns might act as a pre-pattern for initial symmetry breaking, and make downstream patterning more reproducible. 

Therefore, in the context of symmetry-breaking events and patterns in developmental biology, it is crucial to consider mechanisms beyond traditional Turing dispersion relations and include Hopf, Turing I-Hopf, and simple unstable systems as potential explanations for biological patterning.


There is evidence for the above mentioned ephemeral patterns, i.e. transient patterns occurring as the system transitions from Turing to stable states, to be relevant for developmental biology. If the patterned state has a longer life time than down stream gene expression, then they suffice in temporarily activating the necessary genes to produce a permanent phenotype.
For instance, digit formation or hair follicle development requires periodic patterns only at specific times when a hormone is produced to generate fingers or hairs~\parencite{raspopovic2014digit,glover2023developmental}. 
Understanding how interactions between different steady states influence pattern formation is essential, as multistability is widespread and plays a critical role in biological systems~\parencite{laurent1999multistability}. In the authors' view, dynamics is more important than steady-state patterns for embryonic development.


An important finding of our work is that absorbing boundary conditions might be part of the solution to the lack of Turing patterning robustness. A significant proportion of cases fall into the top diagonal of the confusion matrix (Fig.~\ref{sup_fig5}A), suggesting that absorbing boundary conditions might enhance patterning robustness.
Specifically, absorbing boundaries can induce spatial patterning by creating non-stationary or stationary periodic patterns from spatially homogeneous patterns (Fig.~\ref{fig:boundariesgrowth}B,C). Absorbing boundaries (and growth) were also used in the experimental setup in ~\parencite{Oliver2023}, which produced a high number of patterns despite the fragility of the Turing mechanism. However, adding absorbing boundaries can also disrupt spatial patterns, as shown in the Sankey diagram (Fig.~\ref{fig:boundariesgrowth}A). Hence, each effect is a two-sided sword in both positvely and negatively affecting pattern formation.

In contrast to absorbing boundaries, introducing growth did not seem to improve robustness for pattern formation as more cases ended up at the bottom of the diagonal in the confusion matrix (Fig.~\ref{sup_fig5}B).
This finding contradicts some literature on growth-induced Turing patterns ~\parencite{gaffney2010}.
The discrepancy may arise from differences in growth rates and types of growth (e.g. exponential or logistic).
Using varying growth rates might enhance robustness or lead to different pattern types, such as interior stripe growth instead of the outer stripe addition shown in Fig. ~\ref{fig:boundariesgrowth}D-bottom~\parencite{konow2019turing}. Note that the growth rate used in this study is slower than the experimental growth rate in ~\parencite{Oliver2023} as long times were simulated to reach convergence in the non-growing reflective boundaries case. Future research may want to test robustness for specific growth rates within an experimental system. Insights into which growth rates most effectively promote pattern formation could be valuable for optimizing experiments.
Future work may also aim to develop a unified classification method that can interpret spatio-temporal patterns across different types of boundaries and both non-growing and growing domains. Such a method would produce comparable outputs to better understand the effects of growth and different boundary conditions on pattern formation.
Although limited, our approach still identifies interesting cases where growth or boundaries influence patterning.


In conclusion, by including the realistic effects of multistability, growth, and boundary conditions, Turing patterns begin to bridge to biological applications. In particular, absorbing boundary conditions, by setting up chemical gradients reminiscent of Wolpert's French flag model~\parencite{wolpert1969positional}, improve structural robustness of the Turing parameter space significantly. As those effects generally require the full nonlinear system and hence numerical treatment, this underscores the importance of looking beyond common linear stability analysis. Applications in engineering Turing patterns with synthetic circuits are a promising way forward.
