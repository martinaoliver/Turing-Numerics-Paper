\section{Supplementary Material}
\newcommand{\beginsupplement}{%
    \setcounter{table}{0}
    \renewcommand{\thetable}{S\arabic{table}}%
    \setcounter{figure}{0}
    \renewcommand{\thefigure}{S\arabic{figure}}%
}
\beginsupplement

\begin{figure}[!ht]
    \center
    \includegraphics[width=0.3\textwidth]{figures/stencils}

    \caption{\textbf{Crank-Nicolson (CN) algorithm for numerical solution}. A stencil is a geometric representation with nodes and edges that represents the points of interest for the numerical approximation. The points of interest, which are the ones present in the equations, are shown in green. Labels $j$ and $n$ are the current space and time points. The CN stencil has one spatial dimension and one temporal dimension, with axes labels time and space ($x$). }   \label{sup_fig1}
\end{figure}


\begin{figure}[!h]
    \includegraphics[width=1\textwidth]{figures/no_growth_classification}

    \caption{\textbf{Decision tree for pattern classification in non-growing domains with reflective boundaries}. A decision tree is based on two layers: spatial homogeneity and convergence. The numerical solutions for the four different pattern outcomes including  homogeneous, temporal oscillator, non-stationary, and stationary pattern as shown below.}
    \label{sup_fig2}
\end{figure}


\begin{figure}[!h]
    \includegraphics[width=1\textwidth]{figures/growth_classification}

    \caption{\textbf{Decision tree for pattern classification in non-growing domains with reflective boundaries}. A decision tree is based on two layers: spatial homogeneity and convergence. The numerical solutions for the four different pattern outcomes including  homogeneous, temporal oscillator, non-stationary, and stationary pattern as shown below.}
    \label{sup_fig3}
\end{figure}


\begin{figure}[!h]
    \includegraphics[width=1\textwidth]{figures/multistability_leftovers}

    \caption{\textbf{Other types of multistability dynamics.} \textbf{(A)} Unstable state converges into Turing. \textbf{(B)} Unstable state produces pattern, while Turing state loses pattern. \textbf{(C)} Multistability disrupts all patterns. \textbf{(D)} Turing I-Hopf state attracts unstable state and generates pattern.}

    \label{sup_fig4}
\end{figure}

